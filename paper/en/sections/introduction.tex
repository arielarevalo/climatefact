\section{Introduction}
Climate change is one of the most pressing challenges facing humanity today. Its growing impact on ecosystems, the global economy, and people's daily lives makes this phenomenon a priority for governments, scientists, and citizens alike. According to the IPCC report, the human influence on the climate system is undeniable and is directly linked to historical greenhouse gas emissions \cite{ipcc2022}. However, despite this scientific consensus, climate change continues to be the subject of misinformation, with erroneous claims that distort public perception and hinder the implementation of effective policies.

In this context, various international figures such as Ant\'{o}nio Guterres, Secretary-General of the United Nations, have made urgent calls to action, highlighting the need to take drastic measures to prevent even more severe consequences \cite{planelles2022}. Furthermore, experts such as Fern\'{a}ndez Muerza (2022) have emphasized that effective communication about climate change is essential for citizens to understand the magnitude of the problem and make informed, responsible decisions \cite{fernandez2022}. However, when information sources are unreliable or contain incorrect claims, the risk of making misguided decisions increases significantly. Therefore, it is crucial to have tools that help filter and verify available information. This raises the question: Is it possible to build an AI agent that detects inconsistencies in climate change claims using retrieval-augmented generation techniques?

The general objective of this work is to build a computational tool capable of identifying and describing inconsistencies in climate change-related claims through automatic fact-checking techniques. This proposal falls within contemporary fact-checking approaches applied to scientific and environmental topics.

To achieve this purpose, the following specific objectives are proposed:

\begin{enumerate}
    \item Build a reliable knowledge base on climate change from official documents issued by recognized institutions, such as the IPCC.

    \item Design an automated workflow that retrieves relevant evidence from the knowledge base to contrast it against external claims.

    \item Develop a mechanism for generating and evaluating inconsistencies that details the differences between claims and retrieved evidence, as well as a validation process to measure its effectiveness.
\end{enumerate}

To this end, a methodology based on Retrieval-Augmented Generation (RAG) will be employed, combining generative models with semantic retrieval techniques. This approach will allow contrasting claims against authoritative technical content, providing precise and efficient validation.

The development of this system not only seeks to contribute to semantic verification in the context of climate change but also to strengthen initiatives against environmental misinformation through the use of artificial intelligence. By providing a reliable mechanism for validating climate change claims, this intelligent agent can facilitate more informed decision-making in both academic settings and public policy formulation. Additionally, it could be a useful tool for governments, non-governmental organizations, and companies dealing with climate change, improving the quality and effectiveness of responses to this global challenge.

The second section of the paper presents related work, reviewing previous research related to automatic claim verification and the use of language models in the context of climate change. Then, the conceptual framework is presented, detailing the theoretical and technical foundations supporting the proposal. The methodology section describes the system design and development, from knowledge base construction to RAG model application. Subsequently, the results obtained from the system implementation are presented, and finally, a discussion is included analyzing the findings, model limitations, and potential future applications.

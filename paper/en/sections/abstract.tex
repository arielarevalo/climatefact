\begin{abstract}
This project presents a fact-checking workflow called ClimateFact that aims to evaluate climate change-related statements using trustworthy scientific knowledge bases. Motivated by the prevalence of climate misinformation and the lack of accessible, domain-specific verification tools, ClimateFact uses Retrieval-Augmented Generation (RAG) to identify and explain inconsistencies between claims and authoritative sources such as the IPCC reports.

The workflow generates a knowledge base using embeddings and regex-based indexing informed by domain-specific ontologies. It then retrieves relevant passages using a multi-step retrieval and re-ranking pipeline, followed by a natural language inference (NLI) model to assess the factual alignment of claims.

Finally, an LLM generates grounded explanations marking contradictions with references. The methodology includes reproducible evaluation pipelines to measure retrieval accuracy, factual consistency, and explanation quality.
\end{abstract}

\begin{IEEEkeywords}
fact-checking, climate change, RAG, NLI, retrieval-augmented generation
\end{IEEEkeywords}

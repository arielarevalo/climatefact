% Diagrama 3: Pipeline de Construcción de la Base de Conocimiento
\begin{tikzpicture}[node distance=1.1cm]
  % Cadena lineal
  \node[io] (pdf) {PDF IPCC AR6};
  \node[process, below=of pdf] (extract) {Extracción de PDF};
  \node[process, below=of extract] (coref) {Resolución de correferencias};
  \node[process, below=of coref] (sent) {Segmentación de oraciones};

  % Bifurcación en dos ramas
  \node[process, below left=1.1cm and 1.0cm of sent] (vec)
    {Vectorización semántica};
  \node[process, below right=1.1cm and 1.0cm of sent] (cidx)
    {Construcción del índice de conceptos};

  % Almacenes de datos — más grandes para legibilidad
  \node[datastore, below=0.9cm of vec] (passages)
    {\texttt{passages.jsonl}};
  \node[datastore, below=0.9cm of cidx] (index)
    {\texttt{concept\_index.json}};

  % Flechas — cadena lineal
  \draw[arrow] (pdf) -- (extract);
  \draw[arrow] (extract) -- (coref);
  \draw[arrow] (coref) -- (sent);

  % Flechas de bifurcación
  \draw[arrow] (sent) -- (vec);
  \draw[arrow] (sent) -- (cidx);

  % Hacia almacenes de datos
  \draw[arrow] (vec) -- (passages);
  \draw[arrow] (cidx) -- (index);

  % Anotaciones
  \node[annot, right=0.3cm of extract] {MinerU};
  \node[annot, right=0.3cm of coref] {AllenNLP / SpanBERT-large};
  \node[annot, left=0.15cm of vec, anchor=east] {text-embedding-3-small};
  \node[annot, right=0.15cm of cidx, anchor=west] {regex + spaCy + NLTK};
\end{tikzpicture}

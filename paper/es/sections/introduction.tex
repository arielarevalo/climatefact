\section{Introducción}
El cambio climático es uno de los desafíos más urgentes que enfrenta la humanidad en la actualidad. Su impacto sobre los ecosistemas, la economía global y la vida cotidiana de las personas es creciente, lo que convierte este fenómeno en una prioridad para gobiernos, científicos y ciudadanos por igual. Según el informe del IPCC, la influencia humana sobre el sistema climático es innegable, y está directamente relacionada con las emisiones históricas de gases de efecto invernadero \cite{ipcc2022}. Sin embargo, a pesar de este consenso científico, el cambio climático continúa siendo objeto de desinformación, con afirmaciones erróneas que distorsionan la percepción pública y dificultan la implementación de políticas efectivas.

En este contexto, diversas figuras internacionales como António Guterres, Secretario General de las Naciones Unidas, han realizado llamados urgentes a la acción, destacando la necesidad de tomar medidas drásticas para evitar consecuencias aún más graves \cite{planelles2022}. Además, expertos como Fernández Muerza (2022) han enfatizado que la comunicación efectiva sobre el cambio climático es esencial para que la ciudadanía comprenda la magnitud del problema y pueda tomar decisiones informadas y responsables \cite{fernandez2022}. Sin embargo, cuando las fuentes de información no son confiables o contienen afirmaciones incorrectas, el riesgo de tomar decisiones equivocadas aumenta significativamente. Por lo tanto, resulta crucial contar con herramientas que ayuden a filtrar y verificar la información disponible. A partir de este contexto surge la pregunta, ¿Es posible construir un agente IA que detecte inconsistencias en afirmaciones sobre el cambio climático con el uso de técnicas de recuperación y generación aumentada?

El objetivo general de este trabajo es construir una herramienta computacional capaz de identificar y describir inconsistencias en afirmaciones relacionadas con el cambio climático, mediante técnicas de verificación automática de hechos. Esta propuesta se enmarca dentro de los enfoques contemporáneos de fact-checking aplicados a temas científicos y ambientales.

Para alcanzar este propósito, se plantean los siguientes objetivos específicos:

\begin{enumerate}
    \item Construir una base de conocimiento confiable sobre el cambio climático, a partir de documentos oficiales emitidos por instituciones reconocidas, como el IPCC.

    \item Diseñar un flujo de trabajo automatizado que permita recuperar evidencia relevante desde la base de conocimiento, con el fin de contrastarla contra afirmaciones externas.

    \item Elaborar un mecanismo de generación y evaluación de inconsistencias, que detalle las diferencias entre afirmaciones y evidencia recuperada, así como un proceso de validación que permita medir su efectividad.
\end{enumerate}

Para ello, se empleará una metodología basada en Retrieval-Augmented Generation (RAG), que combina modelos generativos con técnicas de recuperación semántica. Este enfoque permitirá contrastar afirmaciones con contenido técnico autorizado, proporcionando una validación precisa y eficiente.

El desarrollo de este sistema no solo busca contribuir a la verificación semántica en el contexto del cambio climático, sino también a fortalecer las iniciativas contra la desinformación ambiental mediante el uso de inteligencia artificial. Al proporcionar un mecanismo fiable para la validación de afirmaciones sobre el cambio climático, este agente inteligente puede facilitar la toma de decisiones más fundamentadas tanto en el ámbito académico como en la formulación de políticas públicas. Además, podría ser una herramienta útil para gobiernos, organizaciones no gubernamentales y empresas que lidian con el cambio climático, mejorando la calidad y la efectividad de las respuestas a este reto global.

En la segunda sección del trabajo, se presenta el estado del arte, donde se revisan investigaciones previas relacionadas con la verificación automática de afirmaciones y el uso de modelos de lenguaje en el contexto del cambio climático. Luego, se expone el marco metodológico, que detalla los fundamentos teóricos y técnicos que respaldan la propuesta. La sección de metodología describe el diseño y desarrollo del sistema, desde la construcción de la base de conocimiento hasta la aplicación del modelo RAG. Posteriormente, se presentan los resultados obtenidos a partir de la implementación del sistema, y finalmente, se incluye una discusión donde se analizan los hallazgos, las limitaciones del modelo y sus posibles aplicaciones futuras.
